For our project we used the Hardware Description Language (HDL) SystemVerilog. SystemVerilog allows to create a device-independent representation of digital logic and is the de-facto standard for digital design and expecially for verification (together with VHDL). 
In SystemVerilog, the \emph{modules} are the basic building blocks. SystemVerilog design consists in interconnected modules.
Our implementation is based on two new interconnected modules together with the original FPU, properly modified:

\begin{itemize}
\item \emph{lampFPU\_sqrt.sv} is the top module, called from the external component of the FPU; it mainly computes the exponent of the number we want the Square Root (or Inv Square Root) of, manages some special cases (Square Root of NaN, Square Root of negative number, etc.) and instantiates \emph{SquareRootModule.sv}
\item \emph{SquareRootModule.sv} implements Goldschmidt's algorithm for the Mantissa of a bfloat16 number. 
\item Our source code also uses some functions and constants in order to make the design parametric. We imported the already existing package \emph{lampFPU\_pkg.sv} containing some useful parameters. We then added some other values and functions we needed. We highlight that in doing so we greatly accelerated the integration process of our modules with the rest of the FPU.
\item We modified \emph{lampFPU\_top.sv} in order to properly manage the square root operation, adding the required internal wires, instantiating \emph{lampFPU\_sqrt.sv} and binding the inputs and outputs to our module. We assigned the square root to the operation code 10 and to the inverse square root to 11. 
\item The folder Source\_Code also contains other SystemVerilog files already implemented for the bfloat16 FPU. We need these modules because \emph{lampFPU\_top.sv} instantiates them.
\end{itemize}

The Square Root algorithm requires to compute not only the Mantissa, but the exponent too. \emph{lampFPU\_sqrt.sv} and  \emph{SquareRootModule.sv} need to collaborate in order to obtain the correct result. 
\begin{itemize}
\item If the exponent is even, we simply divide it by 2 in the external module \emph{lampFPU\_sqrt.sv}  (so we right shift the exponent).
\item If the exponent is odd, simply right shifting would not produce the right result. In order to compute the correct final value, we make the exponent even by exploiting:
$$F = (-1)^{S}*M*2^{E}  = (-1)^{S}*(M*2)*2^{E-1} $$
We can now divide by 2 the exponent and we inform the internal module that the exponent was odd with the input \emph{is\_exp\_odd\_i}. \\
We then exploit a well-known property of the Square Root:
$$ \sqrt{2*M} = \sqrt{2} * \sqrt{M} $$
So, we actually pass M (and not $2 * M$) to the internal module, but we multiply the result computed by Goldschmidt's algorithm by $\sqrt{2}$ before returning it.\\
The same mathematical proprierties are applied when we want to compute the inverse square root, but in this case, we exploit:
$$ \sqrt{\frac{1}{2*M}} = \frac{1}{\sqrt{2} * \sqrt{M}} $$
\end{itemize} 

\subsection{SquareRootModule.sv}
We firstly describe the SquareRootModule.sv module. \\
In general, \emph{SquareRootModule.sv} takes as input an 8-bits Mantissa M between 1.0 (8b'100000000) and 1.9921875 (8b'11111111) and returns 16 bits representing $\sqrt{M}$ or $\frac{1}{\sqrt{M}}$ (the first bit is implicity 1, as expected by the IEEE 754 standard). The Mantissa length is greater than expected by the bfloat16 standard, but we'll use this enhanched precision in order to properly round the result when needed.\\\\
%----- add module interface ---
The sequential logic of \emph{SquareRootModule.sv} implements a synchronous reset of the module, setting to zero all the internal wires if the input \emph{rst} is 1. Otherwise, it saves the values computed by the combinational logic into internal registers.\\\\
The combinational logic of \emph{SquareRootModule.sv} implements a Finite State Machine with four different states:
\begin{enumerate}
\item Every times it starts, we compute the various values required by the algorithm, but we don't save them untill we're in the right State.
\item In the IDLE state, we initialize our internal variables as described in Section \ref{goldschmidt}, only if the external input \emph{doSqrt\_i} signals to start the computation and we're not in one of the special cases (Square Root of NaN, Square Root of Zero...). To note, we declare the number 3 as 9'b110000000; since our Significant occupies 8 bits and is between 1.0 and 1.9921875, in order to represent the number 3 we need to add an additional bit. 
\item SQRT\_B is the output state; if $R_i$ is 1, we stop the computation, return the result (either the square root or the inverse square root) and signal the end by setting \emph{valid\_next} as 1. If the exponent of the bfloat16 we want the square root of is odd, we first multiply $X_i$ by $\sqrt{2}$ and $Y_i$ by $\frac{1}{\sqrt{2}}$. Goldschmidt's algorithm doesn't yeld a precise LSB, but we return x\_tmp or y\_tmp, which have more precision than our final result. We delegate the rounding to \emph{lampFPU\_top.sv}. On the other hand, if the estimation of the square root is not good enough, we perform another iteration of the algorithm, by updating \emph{b\_next} and moving to the next state.
\item In the SQRT\_R state, we update $R_i$ as requested by the algorithm.
\item In the SQRT\_XY state, we update $X_i$ and $Y_i$ as requested by the algorithm and move back to SQRT\_B.
\end{enumerate}

\subsection{lampFPU\_sqrt.sv}
We now describe the lampFPU\_sqrt.sv module.\\
In general, \emph{lampFPU\_sqrt.sv} module instantiates \emph{SquareRootModule.sv}, computes the final exponent and manages some special cases. \\ 
The square root is a particular operation, since it can't produce neither an Overflow nor an Underflow. We remember that overflow occurs when positive numbers exceed the maximum value or negative numbers exceed the maximum negative value that can be represented. For the square root, we have that:
\begin{itemize}
\item If $X > 1$, then $X > \sqrt{X} > 1$ 
\item If $0 < X < 1$, then $0 < X < \sqrt{X} < 1$
\item If $X < 0$, then $\sqrt{X}$ is NaN
\end{itemize}
This means that if X is a number that can be represented in bfloat16, its square root can always be represented in bfloat16. For this reason, \emph{lampFPU\_sqrt.sv} doesn't need to manage Overflow or Underflow cases.\\ \\
% add why don't need normalization
%add module interface%
The sequential logic of  \emph{lampFPU\_sqrt.sv} resets to zero the module's internal wires and the outputs when the input \emph{rst} is 1. Otherwise, it saves some of the inputs into internal registers and updates the outputs.\\\\
The combinational logic of \emph{lampFPU\_sqrt.sv} firstly checks if we're trying to compute the Square Root or Inverse Square Root of a special case, using the function \emph{FUNC\_calcInfNanZeroResSqrt}. In this case, the output doesn't need to be rounded. We have the following:
\begin{itemize}
\item $\sqrt{+0}$ = $+0$ 
\item $\sqrt{-0}$ = $-0$
\item $\sqrt{NaN}$ = NaN
\item $\sqrt{+ \infty}$  =  $+\infty$
\item $\sqrt{- \infty}$ =  NaN
\item $\sqrt{- X}$ = NaN  (X is any generic number) 
\item $\frac{1}{\sqrt{+0}}$ = NaN
\item $\frac{1}{\sqrt{-0}}$ = NaN
\item $\frac{1}{\sqrt{NaN}}$ = NaN (Inf???)
\item $\frac{1}{\sqrt{+\infty}}$ = $+0$
\item $\frac{1}{\sqrt{-\infty}}$ = NaN
\item $\frac{1}{\sqrt{-X}}$ = NaN  (X is any generic number) 
\end{itemize}

If we are not in a special case, we compute the exponent and wait until the internal module computes the square root or inverse square root of the Mantissa (we uses  \emph{srm\_valid} for it) and then we return the final value as outputs. To note, the signal \emph{f\_res\_o} has a dimention of 12 bits, and it needs to be rounded. The rounding is delegated to the top module, and is based on three bits:
\begin{itemize}
\item The Ground bit is the bit immediately on the left of the Mantissa LSB, so it's the 4\textsuperscript{th} bit of \emph{srm\_res}.
\item The Round bit is on the left of the Ground bit, so it's the 3\textsuperscript{rd} bit of \emph{srm\_res}.
\item The Sticky bit is computed as the logical or of the three remaining bits, from the 2\textsuperscript{nd} to the 0\textsuperscript{th} bit of \emph{srm\_res}. 
\end{itemize}

We inform the top module with the signal \emph{isToRound}; the rounding is performed using the function \emph{FUNC\_rndToNearestEven} which implements the following rules:
\begin{itemize}
\item If GRS = 0XX, then round down
\item If GRS = 100, then round up if mantissa LSB is 1, otherwise round down
\item Any other case, round up
\end{itemize} 

\clearpage
